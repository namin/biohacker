\section{Results} 
To test the ability of {\tt BIOHACKER} to debug in real data, we
exported the EcoCyc\cite{ecocyc}\cite{romero2001} representation of the {\em E. coli}
metabolic network into the {\tt BIOHACKER} domain data format, generating
929 {\tt reaction}s, 836 {\tt enzyme}s, and 251 {\tt pathway} premises.

To ensure that our model would grow on at least one nutrient media, we generated an {\em in silico} rich medium
set by taking the union of all proper reactants of each pathway in the model, as shown in Listing~\ref{listing:richMedia}

\begin{lstlisting}[label={listing:richMedia},caption={{\em E. coli} rich media}]
(defun make-rich-media (pwy-list filter-p)
  `(experiment 
    growth 
    (nutrients 
     ,@(remove-duplicates 
	(loop for pwy in pwy-list
	   when (funcall filter-p pwy)
	   append (multiple-value-bind
			(all-reactants 
			 proper-reactants 
			 all-products 
			 proper-products)
		      (substrates-of-pathway pwy)
		    proper-reactants)))) (OFF)))
\end{lstlisting}

We then selected as our essential compounds the full set of amino acids, nucleic acids, cytoplasmic membrane components,
outer membrane components, and cell wall components as shown in Listing~\ref{listing:essentialCompounds}

\begin{lstlisting}[label={listing:essentialCompounds},caption={{\em E. coli} essential compounds}]
(setq *amino-acids*
  '(L-ALPHA-ALANINE ARG ASN L-ASPARTATE CYS GLN 
    GLT GLY HIS ILE LEU LYS MET PHE PRO SER THR 
    TRP TYR VAL))

(setq *dna-and-rna*
  '(DATP TTP DGTP DCTP ATP UTP GTP CTP))

(setq *cytoplasmic-membrane*
  '(L-1-PHOSPHATIDYL-ETHANOLAMINE CARDIOLIPIN 
    L-1-PHOSPHATIDYL-GLYCEROL))

(setq *outer-membrane* '(C6))
(setq *cell-wall*
  '(BISOHMYR-GLC ADP-L-GLYCERO-D-MANNO-HEPTOSE 
    KDO UDP-GLUCOSE UDP-GALACTOSE DTDP-RHAMNOSE 
    GDP-MANNOSE N-ACETYL-D-GLUCOSAMINE))
\end{lstlisting}

Surprisingly, even with this rich media, we were still unable to produce compound {\tt C6} (lipid intermediate II)

By manual checking, we discovered that the reason for this is that an upstream reaction, {\tt UDPNACETYLMURAMATEDEHYDROG-RXN }, in the Peptidoglycan biosynthesis pathway was reversed, making the entire chain of downstream metabolites unproduceable. 

Once this bug was discovered,  we added a new rule to our set of strategies such that
 if a metabolite is unproducable, the system tries to reverse
reactions with unknown reversibilities to see if that will
solve the problem.  

Interestingly, it has been shown that for many metabolic reconstructions,
that ''the dominant flow restoration mechanism is directionality
reversals of existing reactions in the respective
models''\cite{kumar2007}

Once we debugged this problem, we wished to reduce the rich medium to a minimal nutrient set that was still
capable of producing growth. By applying the abduction mechanism of the underlying TMS, we were able to make the following
minimal nutrient set prediciton as shown in Listing~\ref{listing:minimalNutrientSet}.
\begin{lstlisting}[label={listing:minimalNutrientSet},caption={{\em E. coli} predicted minimal nutrient set}]
(GLN ASN THR LEU VAL ILE TRP PHE TYR CYS MET 
     LYS GLY ARG HIS PRO UDP-GALACTOSE 
     UDP-GLUCOSE RIBULOSE-5P 
     ADP-L-GLYCERO-D-MANNO-HEPTOSE GLC-1-P 
     TTP FRUCTOSE-6P 3-OHMYRISTOYL-ACP GLT 
     NADPH PHOSPHO-ENOL-PYRUVATE 
     MESO-DIAMINOPIMELATE L-ALPHA-ALANINE 
     D-ALANINE UDP-N-ACETYL-D-GLUCOSAMINE 
     UNDECAPRENYL-P GTP GLYCEROL-3P SER
     CDPDIACYLGLYCEROL METHYLENE-THF UTP 
     CTP UDP |Red-Glutaredoxins|
     |Red-Thioredoxin| |Reduced-flavodoxins| 
     CDP ATP WATER
     DIACETYLCHITOBIOSE-6-PHOSPHATE)
\end{lstlisting}
This minimal nutrient set allowed us to focus on reactions and pathways that could produce these compounds starting
with experimentally determined nutrient sets, such as Middlebrook growth medium L9, as described in \cite{joyce2006}.


Finally, we then converted 13750 conditional gene essentiality experiments performed by
Covert et al\cite{covert2004} and for each condition, we recorded
whether the {\em E. coli} model's predictions of growth were
consistent or inconsistent with the experimental results. as shown in Listing~\ref{listing:biolog-experiment}.

\begin{lstlisting}[label={listing:biolog-experiments},caption={{\em E. coli} conditional essentiality experiments}]
(experiment in_vivo_EG11074_ala-D_nh4 
	    ( CARBON-DIOXIDE PROTON WATER 
			     AMMONIUM 
			     SULFATE 
			     D-ALANINE |Pi| 
			     OXYGEN-MOLECULE )
   :growth?  T
   :essential-compounds 
   ( L-ALPHA-ALANINE ARG ASN L-ASPARTATE CYS
     GLN GLT GLY HIS ILE LEU LYS MET PHE PRO 
     SER THR TRP TYR VAL DATP TTP DGTP DCTP 
     ATP UTP GTP CTP 
     L-1-PHOSPHATIDYL-ETHANOLAMINE CARDIOLIPIN
     L-1-PHOSPHATIDYL-GLYCEROL C6 BISOHMYR-GLC
     ADP-L-GLYCERO-D-MANNO-HEPTOSE KDO 
     UDP-GLUCOSE  UDP-GALACTOSE DTDP-RHAMNOSE 
     GDP-MANNOSE N-ACETYL-D-GLUCOSAMINE )

   :knock-outs ( EG11074 )
   :knock-ins Nil
   :toxins Nil
   :bootstrap-compounds Nil)

\end{lstlisting}

As we discovered, most of the model predictions were false negatives
due to the existence of uninstantiated generic reactions. Inspired by
the symbolic computational approach to infer novel biochemical
knowledge described in \cite{McShan2004}, we decided to implement the
{\tt METABOLIZER} portion of our project.
